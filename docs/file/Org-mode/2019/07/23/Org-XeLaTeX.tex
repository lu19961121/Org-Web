% Created 2019-07-23 周二 21:32
% Intended LaTeX compiler: xelatex
\documentclass{ctexrep}


\usepackage{ecnu}
\date{\today}
\title{Org-XeLaTeX}
\hypersetup{
 pdfauthor={},
 pdftitle={Org-XeLaTeX},
 pdfkeywords={},
 pdfsubject={},
 pdfcreator={Emacs 26.2 (Org mode 9.2.4)}, 
 pdflang={English}}
\begin{document}

Org-XeLaTeX
\tableofcontents


\section{写作缘由}
\label{sec:org1aedec9}
\LaTeX 的论文排版无可替代,而学校不会专门教排版设计。
针对人文社科类学位论文写作,从基本的文字编辑,数据分析,进度管理,版本控制到最后的排版设计,笔者写下这篇文章,
希望能够帮助大家完成自己的
毕业论文。
默认读者使用 \faWindows  Windows10操作系统,
\faApple  Mac 原理一样,Linux\ldots{}\faLinux 嗯,你应该比我知道的多。
\section{文字编辑}
\label{sec:org1ca4665}
\subsection{Org-mode 安装}
\label{sec:orgd10aa6f}
注:在 buffer 中,C 代表 Ctrl 键, M 代表 Alt 键,RET 代表 Enter,小写字母就是实际字母。

\begin{itemize}
\item 下载\href{https://www.gnu.org/software/emacs/download.html}{Emacs} 26.2,安装最新版 \href{https://orgmode.org/elpa.html}{Org-mode} 9.2.4,官网说的配置文件 \textbf{init file} 需要新建:
\item 打开 \textbf{runemacs.exe} ,C-c C-f 输入 \textbf{.emacs} ,新建完成
\item 按照官网介绍操作,安装完成后,M-x org-version,应该就是与官网一致的版本。
\end{itemize}
\subsection{基本操作}
\label{sec:org8545fea}
参照国外大佬的\href{https://pan.baidu.com/s/1p6CRrnt6c0WrROvLW0BjRA }{视频(提取码:26qz )},主要记 \textbf{快捷键} (输入字母可以识别,但不是最新的,比如大写字母都改小写了),介绍顺序在\href{https://orgmode.org/worg/org-tutorials/org-screencasts/org-mode-google-tech-talk.html}{这儿}。
\subsection{特殊写作环境}
\label{sec:org496d5a9}
写作前,开头加入这段文本:

\begin{verbatim}
# -*- coding: utf-8 -*-
\end{verbatim}

保证你的文字导出时不会乱码。

1.摘要

\begin{verbatim}
#+begin_abstract
摘要测试
#+end_abstract
\end{verbatim}

2.代码

C-c C-,

默认为verbatim环境,
如果要用 minted 宏包,确认你有安装了Python包pygments,建议安装 \textbf{Anaconda} ,然后在 \#+begin
前一行加上

\begin{verbatim}
#+ATTR_LATEX: :options org-latex-minted-options
\end{verbatim}

\section{数据分析}
\label{sec:orgfb26bd8}
SPSS良心14天试用期够用了,如果你经常要写论文,推荐学习一下 \textbf{R} 语言。
\section{{\bfseries\sffamily TODO} 使用 Org-mode 的GTD工作流程}
\label{sec:org118ba18}
\section{版本控制 \href{https://git-scm.com/downloads}{Git} (可选)}
\label{sec:orge0ffa6b}
\href{https://www.liaoxuefeng.com/wiki/896043488029600}{廖雪峰的Git教程}
\section{协同写作}
\label{sec:org410cedb}
C-c C-e 导出utf-8文本,供导师修改(反正排版是最后做的)。
\section{排版设计}
\label{sec:org77c0f8e}
笔者在设计过程中发现Org-mode中用 XeLaTeX 写中文文档很早就有人写配置(添加在 \textbf{.emacs} 文件的)了,
现在笔者添加到 \href{https://github.com/Tom007Cheung/Org-XeLaTeX/blob/master/ox-latex.el}{ox-latex.el} 里,然后编译:M-x byte-compile-file(支持 TAB 键自动补全),重新打开 runemacs.exe (推荐
添加到桌面快捷方式)就可以使用了。
所有文本写完之后,跳到开头,C-c C-e \#,输入latex,应该可以看到一下信息:

\begin{verbatim}
#+latex_class: ctexrep
#+latex_class_options:<默认为[12pt, a4paper],可以自己设置>
#+latex_header:\usepackage{学校 \LaTeX 模板样式(.sty结尾),我用的 ecnu}
#+latex_header_extra:
#+description:
#+keywords:
#+subtitle:
#+title: <默认显示你的文件名>
#+latex_compiler: xelatex
#+date: \today
\end{verbatim}

如果有学校封面,title,date可以删掉。

表格,图片一律使用图片导入:C-c C-l file,选择你的本地图片路径即可。
\subsection{安装 \href{https://zhuanlan.zhihu.com/p/64555335}{TeXLive}(强烈推荐)2019}
\label{sec:org0d5f60d}
安装完成后,打开你的 \textbf{.org} 文件,C-c C-e l o,
即可打开编译好的 \textbf{pdf} 文件,当然可以自己改一下
\textbf{.tex} 文件,这样排版工作量就会少很多。
\section{参考链接}
\label{sec:orge3ab2c7}
\subsection{基于 gbt-7714-2015 格式参考文献编译}
\label{sec:org593eb55}
\subsubsection{\href{https://github.com/jkitchin/org-ref}{org ref} 插件}
\label{sec:org7b1a9e0}

[1] \url{https://www.reddit.com/r/emacs/comments/4k1lp2/noob\_question\_how\_to\_set\_locales\_and\_encoding\_for/}

[2] \url{https://www.cnblogs.com/wangkangluo1/archive/2012/02/04/2337705.html}

[3] \url{http://www.cnblogs.com/visayafan/archive/2012/06/16/2552023.html}

[4] \url{https://xiaoguo.net/wiki/org-mode-book.html}

[5] \url{https://orgmode.org/manual/index.html\#SEC\_Contents}

[6] \url{https://orgmode.org/worg/org-tutorials/org-screencasts/org-mode-google-tech-talk.html\#sec-2}

[7] \url{https://zhuanlan.zhihu.com/p/64555335}

[8] \url{https://github.com/jkitchin/org-ref/blob/master/org-ref.org}
\end{document}
